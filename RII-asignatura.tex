
\section{Objetivos}
\subsection{De la carrera}
Según el documento fundamental de la Tecnicatura, el Técnico Superior en Administración de Sistemas y Software Libre estará capacitado para:
\begin{itemize}
	\item Desarrollar actividades de administración de infraestructura. Comprendiendo la administración de sistemas, redes y los distintos componentes que forman la
infraestructura de tecnología de una institución, ya sea pública o privada.
	\item Aportar criterios básicos para la toma de decisiones relativas a la adopción de nuevas tecnologías libres.
	\item Desempeñarse como soporte técnico, solucionando problemas afines por medio de la comunicación con comunidades de Software Libre, empresas y desarrolladores de
software.
	\item Realizar tareas de trabajo en modo colaborativo, intrínseco al uso de tecnologías libres.
	\item Comprender y adoptar el estado del arte local, nacional y regional en lo referente a implementación de tecnologías libres. Tanto en los aspectos técnicos como legales.
\end{itemize}
\subsection{De la asignatura}

\begin{itemize}
	\item Actualizar y ampliar el instrumental de trabajo en ambientes de redes.
\end{itemize}


\section{Cursado}
\begin{itemize}
	\item Cuatrimestral de 16 semanas, 4 horas semanales, 64 horas totales
	\item Clases teórico-prácticas presenciales
	\item Promocionable con trabajos prácticos
\end{itemize}


\section {Contenidos}
\subsection{Contenidos mínimos}
\begin{itemize} 
	\item Switching. 
	\item Redes Privadas Virtuales. 
	\item Balance de carga y Alta Disponibilidad en redes. 
\end{itemize}


\subsection {Programa}
\begin{enumerate}
	\item Switching
		\begin{itemize}
			\item Red Ethernet, dominio de colisión y dominio de Broadcast
			\item Bridges, switches, arquitectura y funcionamiento. 
			\item VLANs, trunking
		\end{itemize}
	\item Redes Privadas Virtuales
		\begin{itemize}
			\item Encapsulamiento
			\item VPN de nivel 2 y de nivel 3
			\item Configuración y administración de OpenVPN
			\item Configuración de ruteo y de firewalling en VPN
		\end{itemize}
	\item Alta Disponibilidad y Balance de carga en Redes
		\begin{itemize}
			\item Arquitecturas redundantes 
			\item Bonding y modos de configuración
			\item Protocolo STP 802.1d
			\item Balance de carga, Ruteo por políticas
%			\item Clustering de LB 
		\end{itemize}

\end{enumerate}

\section {Bibliografía inicial}
\begin{itemize}
\item [1]C., Zimmerman, Joann Spurgeon, Ethernet switches. Sebastopol, CA: O’Reilly Media, 2013.
\item [2]M. Feilner, A. Bämler, N. Graf, and M. Heller, Open VPN building and integrating virtual private networks: learn how to build secure VPNs using this powerful open source application. Birmingham, U.K.: Packt Pub., 2006.


\end{itemize}
