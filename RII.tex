\documentclass[11pt,a4paper]{article}
\usepackage[utf8x]{inputenc}
\usepackage[T1]{fontenc}
\usepackage[spanish]{babel}
\usepackage{amsmath}
\usepackage{amssymb,amsfonts,textcomp}
\usepackage{xcolor}
\usepackage{array}
\usepackage{multirow}
\usepackage{hhline}
\PassOptionsToPackage{hyphens}{url}\usepackage{hyperref}
\usepackage{float}
\usepackage{xkeyval}
\usepackage[pdftex]{graphicx}
\usepackage[yyyymmdd,hhmmss]{datetime}
\usepackage{appendix}
\usepackage{listings}
\usepackage{wasysym}
%\usepackage[centering,width=8.5in,height=11in]{geometry}
\usepackage[centering,a4paper]{geometry}
\usepackage{pstricks}
\usepackage{tikz}
\usetikzlibrary{calc, backgrounds}
\usepackage{pdfpages}
\definecolor{dkgreen}{rgb}{0,0.6,0}
\definecolor{gray}{rgb}{0.5,0.5,0.5}
\definecolor{mauve}{rgb}{0.58,0,0.82}
\definecolor{lstbackground}{rgb}{0.95,0.95,0.95}
%\lstset{frame=tb,
%	backgroundcolor=\color{lstbackground},
%  language=Bash,
%  aboveskip=3mm,
%  belowskip=3mm,
%  showstringspaces=false,
%  columns=flexible,
%  basicstyle={\small\ttfamily},
%  numbers=none,
%  numberstyle=\tiny\color{gray},
%  keywordstyle=\color{blue},
%  commentstyle=\color{dkgreen},
%  stringstyle=\color{mauve},
%  breaklines=true,
%  breakatwhitespace=true
%  tabsize=3
%}

\lstset{frame=tb,
	backgroundcolor=\color{lstbackground},
%  language=Bash,
  aboveskip=3mm,
  belowskip=3mm,
  showstringspaces=false,
  columns=flexible,
  basicstyle={\small\ttfamily},
  numbers=none,
%  numberstyle=\tiny\color{gray},
%  keywordstyle=\color{blue},
%  commentstyle=\color{dkgreen},
%  stringstyle=\color{mauve},
  breaklines=true,
  breakatwhitespace=true
  tabsize=4
}

\usepackage{verbatim}
\begin{comment}
	\hypersetup{
		pdftex, 
		colorlinks=true, 
		linkcolor=blue, 
		citecolor=blue, 
		filecolor=blue, 
		urlcolor=blue, 
	pdftitle={Software Libre}, 
	pdfauthor={Eduardo Grosclaude}, 
	pdfsubject={Documento de la materia Software Libre}, 
	pdfkeywords={Software Libre, Tecnicatura en Administración de Sistemas y Software Libre, Universidad Nacional del Comahue}
	}
\end{comment}	

%\addto\captionsspanish {%
%	\def\appendixname{Apéndices}
%}
% Outline numbering
\setcounter{secnumdepth}{1}
% Reset section numbering between parts
\makeatletter
\@addtoreset{section}{part}
\makeatother  
% List styles
\newcommand\liststyleLi{%
\renewcommand\labelitemi{\tiny${\blacksquare}$}
\renewcommand\labelitemii{\tiny${\square}$}
\renewcommand\labelitemiii{\tiny${\circ}$}
\renewcommand\labelitemiv{\tiny${\circ}$}
}
\newcommand\liststyleLii{%
\renewcommand\labelitemi{{\textbullet}}
\renewcommand\labelitemii{${\circ}$}
\renewcommand\labelitemiii{${\blacksquare}$}
\renewcommand\labelitemiv{{\textbullet}}
}
\newcommand\liststyleLiii{%
\renewcommand\labelitemi{{\textbullet}}
\renewcommand\labelitemii{${\circ}$}
\renewcommand\labelitemiii{${\blacksquare}$}
\renewcommand\labelitemiv{{\textbullet}}
}

\liststyleLi

% Page layout (geometry)
\setlength\voffset{-1in}
\setlength\hoffset{-1in}
\setlength\topmargin{2cm}
\setlength\oddsidemargin{2cm}
\setlength\textheight{23.246668cm}
\setlength\textwidth{17.006cm}
\setlength\footskip{26.144882pt}
\setlength\headheight{1.016cm}
\setlength\headsep{0.508cm}
% Footnote rule
\setlength{\skip\footins}{0.119cm}
\renewcommand\footnoterule{\vspace*{-0.018cm}\setlength\leftskip{0pt}\setlength\rightskip{0pt plus 1fil}\noindent\textcolor{black}{\rule{0.25\columnwidth}{0.018cm}}\vspace*{0.101cm}}
% Pages styles
\makeatletter
\newcommand\ps@Standard{
  \renewcommand\@oddhead{{\raggedleft Cabecera \ } {\raggedright \thepage{}}}
  \renewcommand\@evenhead{\@oddhead}
  \renewcommand\@oddfoot{}
  \renewcommand\@evenfoot{\@oddfoot}
  \renewcommand\thepage{\arabic{page}}
}
\makeatother
% \pagestyle{Standard}
\usepackage{fancyhdr}
%\usepackage{libertine}

\usepackage{lmodern}
\renewcommand*\familydefault{\sfdefault} %% Only if the base font of the document is to be sans serif


% \renewcommand*\familydefault{\sfdefault}
\pagestyle{fancy}
% footnotes configuration
\makeatletter
\renewcommand\thefootnote{\arabic{footnote}}
\makeatother
\title{Redes II}
\author{Eduardo Grosclaude}
\date{2014-08-13}
\usepackage{graphicx}

\usepackage{xkeyval}
\usepackage{pifont}
\usepackage{xcolor}
\newcommand{\revisar}[1]{{\color{red}[#1]}}



%\usepackage[hundred]{vrsion}

\newcommand{\borrador}{
\revisar{\today, \currenttime  -  Material en preparación, se ruega no imprimir mientras aparezca esta nota}
}


%\newcommand{\nota}[1]{{\color{red}[#1]}}
%\newcommand{\revisar}[1]{}
\newcommand{\nota}[1]{}

\newcommand{\nonota}[1]{#1}

\newcommand{\quotes}[1]{``#1''}

   
\newcommand{\shade}[1]{\textcolor{black!50}{#1}}

% ancho opcional, por defecto 15cm
% \figura{copyleft}{Símbolo de Copyleft}{copyleft.png}
% \figura[6]{copyleft}{Símbolo de Copyleft}{copyleft.png}
\newcommand{\figura}[4][15]{
 \begin{figure}[htbp] 
 \centering 
 \includegraphics[width=#1cm]{./img/#4} 
 \caption{#3} 
 \label{fig:#2} 
 \end{figure} 
}

\newcommand{\tabla}[4]{
 \begin{table} 
 \centering 
 \small
 \begin{tabular}{#3}
 #4
 \end{tabular}
 \caption{#2}
 \label{tab:#1} 
 \end{table} 
}

\usepackage{mdframed}

\newcommand{\recuadro}[1]{\vspace*{0.2cm}
\begin{minipage}[l]{0.84\textwidth}
\begin{mdframed}
#1
\end{mdframed}
\end{minipage}\vspace*{0.4cm}
}

% \newcommand{\recuadro2}[1]{
% \fcolorbox{black}{grey}{
% \parbox[t]{1.0\linewidth}{ \vspace*{0.4cm}#1\vspace*{0.4cm} } 
% }
% 
\usepackage{sectsty}
%\usepackage[compact]{titlesec} 
\definecolor{bl}{rgb}{0.0,0.2,0.6} 
\allsectionsfont{\color{bl}\scshape\selectfont}


\newcommand{\code}[1]{\lstinline$#1$}
%\newcommand{\code}[1]{\begin{verbatim}{#1}\end{verbatim}}

\hypersetup{pdftex, colorlinks=true, linkcolor=blue, citecolor=blue, filecolor=blue, urlcolor=blue, 
	pdftitle={Redes II}, 
	pdfauthor={Eduardo Grosclaude}, 
	pdfsubject={Documento de la materia Redes II}, 
	pdfkeywords={Redes, Switching, Alta Disponibilidad, Tecnicatura en Administración de Sistemas y Software Libre, Universidad Nacional del Comahue}}

 
% --------------------------------------------------------------------
\begin{document}

\maketitle

\borrador


\abstract { En este escrito se presenta la descripción y material inicial de la asignatura \textbf{Redes II}, para la carrera de Tecnicatura Universitaria en Administración de Sistemas y Software Libre, de la Universidad Nacional del Comahue. 

La materia es electiva, cuatrimestral en modalidad presencial y las clases son de carácter teórico-práctico, desarrolladas en forma colaborativa. Está preparada con los objetivos generales de  \textbf{actualizar y ampliar el instrumental de trabajo en ambientes de redes}. 
  


\newpage
\emph{" "}
\newpage

\tableofcontents

\newpage
\emph{" "}

%----------- P R E S E N T A C I O N  ---------
\newpage
\part {La asignatura}


\section{Objetivos}
\subsection{De la carrera}
Según el documento fundamental de la Tecnicatura, el Técnico Superior en Administración de Sistemas y Software Libre estará capacitado para:
\begin{itemize}
	\item Desarrollar actividades de administración de infraestructura. Comprendiendo la administración de sistemas, redes y los distintos componentes que forman la
infraestructura de tecnología de una institución, ya sea pública o privada.
	\item Aportar criterios básicos para la toma de decisiones relativas a la adopción de nuevas tecnologías libres.
	\item Desempeñarse como soporte técnico, solucionando problemas afines por medio de la comunicación con comunidades de Software Libre, empresas y desarrolladores de
software.
	\item Realizar tareas de trabajo en modo colaborativo, intrínseco al uso de tecnologías libres.
	\item Comprender y adoptar el estado del arte local, nacional y regional en lo referente a implementación de tecnologías libres. Tanto en los aspectos técnicos como legales.
\end{itemize}
\subsection{De la asignatura}

\begin{itemize}
	\item Actualizar y ampliar el instrumental de trabajo en ambientes de redes.
\end{itemize}


\section{Cursado}
\begin{itemize}
	\item Cuatrimestral de 16 semanas, 4 horas semanales, 64 horas totales
	\item Clases teórico-prácticas presenciales
	\item Promocionable con trabajos prácticos
\end{itemize}


\section {Contenidos}
\subsection{Contenidos mínimos}
\begin{itemize} 
	\item Switching. 
	\item Redes Privadas Virtuales. 
	\item Balance de carga y Alta Disponibilidad en redes. 
\end{itemize}


\subsection {Programa}
\begin{enumerate}
	\item Switching
		\begin{itemize}
			\item Red Ethernet, dominio de colisión y dominio de Broadcast
			\item Bridges, switches, arquitectura y funcionamiento. 
			\item VLANs, trunking
		\end{itemize}
	\item Redes Privadas Virtuales
		\begin{itemize}
			\item Encapsulamiento
			\item VPN de nivel 2 y de nivel 3
			\item Configuración y administración de OpenVPN
			\item Configuración de ruteo y de firewalling en VPN
		\end{itemize}
	\item Alta Disponibilidad y Balance de carga en Redes
		\begin{itemize}
			\item Arquitecturas redundantes 
			\item Bonding y modos de configuración
			\item Protocolo STP 802.1d
			\item Balance de carga, Ruteo por políticas
%			\item Clustering de LB 
		\end{itemize}

\end{enumerate}

\section {Bibliografía inicial}
\begin{itemize}
\item C., Zimmerman, Joann Spurgeon, Ethernet switches. Sebastopol, CA: O’Reilly Media, 2013.
\item M. Feilner, A. Bämler, N. Graf, and M. Heller, Open VPN building and integrating virtual private networks. Birmingham, U.K.: Packt Pub., 2006.
\item T. Bourke, Server load balancing, 1st ed. Beijing; Sebastopol, Calif: O’Reilly, 2001.
\end{itemize}

%\input {ASA-eval}
%\input {ASA-cronog}

%----------- M A T E R I A L ---------
\newpage
\part {Switching}



\section{Ethernet}
\begin{enumerate}
\item ¿Qué módulos del kernel Linux están controlando las interfaces de red de su equipo? ¿Cuáles comandos permiten conocer las marcas y modelos de las placas instaladas? ¿Qué diferencia hay entre placas y chipsets? ¿Cómo se puede investigar cuál módulo corresponde a cuál marca y modelo de placa de red? 
\item ¿Qué es el bus PCI y qué son los números de dispositivos asignados? ¿Cuál es la relación entre el bus PCI y los nombres de dispositivos de red en Linux?
\item ¿Qué información aporta el comando ethtool? ¿Qué clases de placas de red soporta? ¿Qué significa \emph{Link detected} en la salida de este comando?
\item ¿Qué información transporta un frame Ethernet? ¿Qué diferencias existen entre los frames que circulan por una Ethernet cableada y una inalámbrica? 
\item ¿Qué diferencias existen entre el diseño de la red Ethernet original, implementada con coaxil, y la que está implementada con cableado en estrella, conectada mediante hubs? ¿Qué diferencias existen entre esta última y una implementada con switches? 
\end{enumerate}


\section{Dominios de colisión y de broadcast}
\label{sub:}
\begin{enumerate}
\item ¿Qué es un bridge, qué configuración lleva, qué función realiza sobre los frames que circulan por la red, y cómo aprende su información de trabajo? 
\item ¿Existen bridges inalámbricos? 
\item ¿A qué se llama segmentación?
\item ¿Un hub es equivalente a un bridge? ¿Un router es equivalente a un bridge? ¿Un switch es equivalente a un bridge? 
\item ¿Qué es un frame de broadcast? ¿Qué es un frame unicast? ¿Cómo se distingue un frame de broadcast? 
\item ¿Qué protocolos utilizan frames de broadcast? ¿Cómo es aprovechado el mecanismo de broadcast por el protocolo ARP? ¿Qué otros usos podría recibir el mecanismo de broadcast?
\item ¿Qué conducta observan los hosts de la red al recibir un frame de broadcast? ¿En qué casos puede resultar esto un problema y por qué? ¿Cuáles son las contramedidas?
\item ¿Qué es un dominio de colisión?  ¿Qué es un dominio de broadcast? ¿Qué tipo de dispositivo es el que delimita la frontera entre dos dominios de colisión?  ¿Qué tipo de dispositivo es el que delimita la frontera entre dos dominios de broadcast? 
\item ¿Qué efecto tiene la introducción de un bridge en un dominio de colisión? El número de equipos en un mismo dominio de colisión ¿se incrementa o se reduce? ¿Qué efecto tiene esto sobre la probabilidad de colisión?
\item ¿Qué diferencia hay entre un paquete de broadcast y un frame de broadcast? Los routers, ¿repiten los frames de broadcasts?
\item En la figura \ref{fig:broadcast}, ¿cuáles son los dominios de colisión y de broadcast que pueden distinguirse? Si el host H1 emite un frame de broadcast, ¿qué otros hosts lo reciben? Si el host H4 emite un broadcast, ¿qué otros hosts lo reciben?
\item En la figura \ref{fig:arbol}, ¿cuáles son los dominios de colisión y de broadcast que pueden distinguirse? Si el host H1 emite un frame de broadcast, ¿qué otros hosts lo reciben? 
\item El concepto de dominio de colisión, ¿sigue siendo aplicable en la actualidad, cuando prácticamente la totalidad de las redes usan bridging en lugar de repetidores? 
\item El uso de bridges, ¿limita el tráfico de broadcast sobre un segmento de red?
 \end{enumerate}
% subsection  (end) Dominios de colisión y de broadcast


\figura[12]{broadcast}{Colisiones y broadcasts}{broadcast.jpg}


\section{Switches}
\label{sub:Switches}

\begin{enumerate}
\item ¿Cuál es la función de un switch? 
\item ¿Cuántos dominios de colisión hay en una red implementada sobre un switch? ¿Cuántos dominios de broadcast hay en una red implementada sobre un switch?
\item ¿En qué consiste el proceso de conmutación de un frame? ¿Qué modos de conmutación son los más usados?
\item ¿Cómo aprende su información de trabajo un switch? ¿Dónde almacena dicha información? ¿Cómo accede a esta información?
\item ¿Qué debe hacer un switch ante un frame dirigido a una estación (\quotes{unicast}) cuya dirección destino figura en sus tablas? 
\item ¿Qué debe hacer un switch ante un frame dirigido a una estación (\quotes{unicast}) que no figura en ninguna de sus tablas? ¿Cómo se llama esta acción?
\item ¿Qué debe hacer un switch ante un frame cuya dirección origen no figura en sus tablas? 
\item ¿Qué debe hacer un switch ante un frame de broadcast? ¿Cómo se llama esta acción?
\item En una topología compuesta por un árbol de tres switches como en la figura \ref{fig:arbol}, ¿cómo aprenden los switches las direcciones de los equipos que no están directamente conectados? ¿Cuántos dominios de broadcast aparecen en dicha figura?
\item ¿En qué orden aparecen las direcciones destino y origen en el frame Ethernet? ¿En qué orden aparecen las direcciones destino y origen en los paquetes IP y en los segmentos TCP o UDP? ¿De qué forma aprovechan los switches la diferencia?

\item En la figura \ref{fig:dominios}, si el host H1 establece conexión TCP con H3, ¿quiénes más pueden ver el tráfico entre H1 y H3? Si el host H1 establece conexión TCP con H4, ¿quiénes más pueden ver el tráfico entre H1 y H4?

\item ¿Cuántas redes aparecen, respectivamente, en las topologías de figuras \ref{fig:broadcast}, \ref{fig:arbol} y \ref{fig:dominios}? ¿Cuántos espacios de direccionamiento IP diferentes habrá en cada topología?

\item ¿Existen switches inalámbricos? ¿En qué se parecen y en qué se diferencian un switch cableado y un punto de acceso inalámbrico? 
\item ¿Qué diferencias existen entre las conexiones de switches en cascada y apiladas (\emph{stacked})? 
\item ¿Qué velocidad suelen tener los ports de un switch? ¿Qué dicen las normas de cableado respecto de la calidad del cable, su longitud y su forma de instalación, para cada caso?
\item Si los ports de un switch tienen una velocidad determinada, ¿tiene sentido conectar a un port un conjunto de hosts capaces de transmitir, en forma agregada, mayor cantidad de bits por segundo que la velocidad admitida por el port? ¿Qué significa \emph{oversubscription}?
\item En una red compartida implementada con hubs, ¿cuál es el ancho de banda obtenido por cada cliente? ¿Y en una red implementada con switches?

\item En la Fig. \ref{fig:velocidades}, supongamos que los hosts H1 y H3 mantienen tráfico entre sí al mismo tiempo que H2 y H4 lo hacen entre sí. ¿Cuál será la velocidad máxima de dicho tráfico si el dispositivo es un hub que funciona a 100Mbps? ¿Y si es un switch que funciona a 100Mbps? 

\item En aquellos switches con ports de diferentes velocidades, ¿cuáles podrían ser los usos de los ports de mayor velocidad?
\item ¿Qué son los dispositivos de red administrables? ¿En qué consiste el protocolo SNMP? ¿Qué características operativas de un switch administrable pueden consultarse o modificarse mediante SNMP? 
\item ¿Qué características administrables de un switch tienen que ver con la gestión de seguridad en redes?
\end{enumerate}
% subsection  (end) Switches
\figura[8]{velocidades}{Competencia por el medio}{velocidades.jpg}



\figura[10]{arbol}{Jerarquía de switches}{arbol.jpg}

\figura[12]{dominios}{Dominios}{dominios.jpg}


\section{Arquitecturas de redundancia}
\begin{enumerate}
\item ¿Qué ventajas aporta la redundancia de enlaces en una red LAN? ¿Qué formas de redundancia se pueden implementar? ¿En qué consiste la tolerancia a fallos?
\item ¿Qué ocurre en una red Ethernet con múltiples caminos entre dos hosts? ¿Qué escenarios de fallo se pueden presentar?  
\item ¿Cómo se puede detectar una tormenta de broadcasts?
\item ¿Qué ventaja ofrece el protocolo IEEE 802.1d (STP)? ¿Cómo es su modo de operación y qué intenta construir? ¿Qué dispositivos soportan 802.1d? 
\item ¿Cómo se elige un dispositivo raíz del árbol STP? ¿Cómo se determinan los caminos al raíz? ¿Qué demora puede tener normalmente la convergencia de STP?
\item ¿Qué datos de configuración o estado, que sean relevantes para 802.1d, y cuya configuración por el administrador sea posible, admiten los bridges y switches? ¿En qué puede influir el administrador de redes sobre la elección del raíz? ¿Cómo se puede aprovechar este hecho?
\item ¿En qué consiste el protocolo RSTP?
\end{enumerate}
% subsubsection  (end) Arquitecturas de redundancia



\section{VLANs}
\label{sub:}
Un diseño clásico de redes de campus consiste en un router que proyecta segmentos de LAN como radios de una estrella (Fig. \ref{fig:bbcolap}). La función de comunicar los radios, que anteriormente era cumplida por el backbone de la red, en este diseño se logra por la conmutación efectuada por el router, por lo cual suele llamarse de backbone colapsado. 

Con este diseño, los dominios de broadcast, y por lo tanto los espacios IP definidos sobre ellos, quedan limitados geográficamente. Si en una zona del campus donde llega un radio de la estrella se necesita ubicar nodos sobre dos dominios de broadcast (porque se desea aislarlos por motivos de seguridad, porque se desea situar equipos sobre dos espacios IP diferentes, o porque se desea limitar la competencia de ambas clases de tráfico por el medio), debe haber dos cableados y deben ocuparse dos bocas del router central. 

\figura[8]{bbcolap}{Backbone colapsado}{backbone-colapsado.jpg}

\figura[12]{vlans}{VLANs}{vlans.jpg}


Con la funcionalidad avanzada de VLANs provista por algunos switches (y definida en el estándar IEEE 802.1Q), el mismo cableado, y el sistema de switches de llegada, puede usarse para conducir dos o más dominios de broadcast. 
 
\begin{itemize}


	\item ¿En qué consiste el diseño de red de campus de backbone colapsado? ¿Qué impacto tiene este diseño sobre la posibilidad de distribuir equipos de diferentes clases sobre una misma región de la red?
	\item 
¿Cuál es la finalidad de definir VLANs en un switch? 

	\item 
¿Cómo se modifica el formato de frame Ethernet para lograr la capacidad de separar los dominios de broadcast al definir VLANs?

	\item ¿Qué debe hacer un switch con VLANs definidas al recibir un frame de broadcast sobre una de sus interfaces? ¿Qué debe hacer con un frame unicast?

	\item En la topología de la figura \ref{fig:vlans}, los tres switches tienen definidas dos VLANs. Los hosts H1, H2 y H4 pertenecen a la VLAN 1, y los hosts H3, H5 y H6 a la VLAN 2. 

	\begin{itemize}
		\item ¿Qué deben hacer los switches con un frame de broadcast recibido desde el host H2?
		\item ¿Qué deben hacer los switches con un frame unicast recibido desde el host H5 y dirigido a H6? ¿Lo mismo, pero desde H5 a H3? ¿Qué diferencia en el formato de los frames existe entre un caso y otro, en cada punto del camino?
		\item ¿Qué condición deben cumplir los ports que interconectan los switches entre sí para poder distribuir las VLANs por toda la topología?
		\item ¿Es posible que una aplicación en el host H5 inicie conexión TCP/IP con una aplicación servidora situada en H2? 
	\end{itemize} 

\item ¿Qué elemento externo es necesario para conectar diferentes VLANs en una misma jerarquía de switches? 
\item ¿Qué son los switches multicapa o \emph{multilayer}? 
\end{itemize} 

% subsection  (end) VLANs


\section {Estudio de caso I}


Una organización desarrolla sus actividades en un campus con dos edificios principales, que alojan las oficinas y talleres. En la organización existen tres áreas principales: Operaciones, Comercialización, e Ingeniería. 

Cada área utiliza un servidor de base de datos principal que es propio del área. La organización cuenta además con un servidor de web y de correo electrónico, un servidor de archivos y un servidor de backups, los tres de uso general para las tres áreas. Se desea que todos los puestos de trabajo puedan acceder además a Internet. 

Una preocupación especial de la organización es darle protección a los servidores y puestos de trabajo de Ingeniería, que no deben ser accedidos desde las demás áreas.
 
La planta del campus y sus edificios, con los principales puntos donde se necesita conectividad, es como indica la Fig. \ref{fig:caso01}. En este diagrama se muestran, con diferentes colores, los puestos de trabajo de cada área. 


\figura[12]{caso01}{Conectar el campus de una organización}{caso01.jpg}


En base a esta información, indique:
\begin{itemize}
	\item Dónde situaría los servidores mencionados.
	\item Dónde ubicaría los elementos de conectividad (switches y routers).
	\item Qué cantidad de puertos debería tener cada uno de estos elementos.
	\item Con qué medios (cobre, fibra, inalámbricos) vincularía los clientes y los elementos de conectividad.
	\item Cómo distribuiría direcciones IP para los clientes y servidores. 
	\item Si utilizaría alguna arquitectura de VLANs, cuál, y por qué. En caso afirmativo, cómo relacionaría las VLANs entre sí, qué enlaces entre switches y routers deben ser de trunking y por qué.
	\item Si utilizaría alguna forma de redundancia, y en caso afirmativo, cuál debe ser el camino normal del tráfico y por qué.
\end{itemize}








\newpage
\part {Redes Privadas Virtuales}
%
\section{Elementos de las VPN}
\begin{itemize}
	\item VPN = \emph{Virtual Private Network}
	\item Un software específico (cliente-servidor) establece una red \emph{virtual} entre las entidades
	\item Sólo aquellos miembros autorizados de la red virtual \emph{privada} pueden utilizarla
	\item \emph{Tunneling} o encapsulamiento de ciertas capas en otras
	\item Encriptación
\end{itemize}


\subsection{Motivación de las VPN}
\begin{itemize}
	\item El software de la organización evoluciona a aplicaciones colaborativas (Groupware, CRM , ERP...)
	\item Trabajo de todos los colaboradores sobre el espacio digital de la organización
	\item Diferentes sedes, o colaboradores (domiciliarios o móviles), en diferentes lugares de la región o del mundo
	\item Soporte universal de Internet en lugar de enlaces arrendados, dedicados
	\item Requerimientos de seguridad
	\item Firewalls y NAT requieren \emph{conduits} o perforaciones para acceder a los servicios de las sedes
 \end{itemize}



\subsection{Modelos de VPN}
\begin{itemize}
	\item Implementadas y administradas por un proveedor de comunicaciones
	\item Implementadas y administradas por el usuario sobre un transporte común (Internet)
\end{itemize}


\subsection{Niveles de implementación}
\begin{itemize}	 

	\item De nivel de Enlace o de capa 2

	\begin{itemize}
		\item Único espacio IP y dominio de broadcast
		\item La VPN trafica frames
		\item Bridging
	\end{itemize} 

	\item De nivel de Red o de capa 3

	\begin{itemize}
		\item  Varias redes con diferentes espacios IP y un servidor VPN que las conecta como backbone 
		\item La VPN trafica paquetes
		\item Ruteo
	\end{itemize}
\end{itemize}


\subsection{Algunas tecnologías}
\begin{itemize}
	\item GRE (RFCs 1701, 1702) Mecanismo general de tunneling, no define encriptación
	\item Protocolos de capa 2: PPTP (Microsoft), L2F, L2TP, L2Sec
	\item Protocolos de capa 3: IPSec
	\item Protocolos de capa 4: SSL/TLS, OpenVPN
\end{itemize}


\figura{vpnbrid} {VPN de capa 2, equivalente a un bridge} {vpn1-bridge.jpg}

\figura{vpnrut} {VPN de capa 3, equivalente a un router} {vpn1-router.jpg}


\section{OpenVPN}

\begin{itemize}
	\item OpenVPN puede funcionar como VPN de capa 2 o de capa 3. 
	\item Proyecto libre compatible con Windows.
	\item No interoperable con IPSec.
	\item Dos procesos, cliente y servidor, corren en diferentes hosts en diferentes redes. 
	\item Cliente y servidor establecen una conexión TCP/IP que utilizan como túnel.
	\item El resto del trabajo lo hace el bridging (en las VPN de capa 2) o el ruteo (de capa 3). 
	\item La arquitectura de OpenVPN (y otras implementaciones de redes virtuales) es un ejemplo claro de uso de interfaces virtuales como los bridges y tun/tap. 
\end{itemize}

\subsection{Soporte necesario para OpenVPN}
\begin{description}
	\item [Tun/Tap] Dispositivos de red virtuales. Trafican unidades de datos entre procesos. 
	\begin{itemize}
		\item Un dispositivo tun (de capa 3) trafica paquetes, mientras que un tap (de capa 2) intercambia frames. 
		\item El propósito básico de Tun/Tap es la creación de túneles.
		\item Un proceso que tiene un dispositivo tun abierto puede escribir un paquete en él.  Del mismo modo, un dispositivo tun que recibe un paquete IP lo envía a un proceso de usuario que tiene abierto ese dispositivo.
		\item Un proceso que tiene un dispositivo tap abierto puede escribir un frame en él.  Del mismo modo, un dispositivo tap que recibe un frame lo envía a un proceso de usuario que tiene abierto ese dispositivo.
	\end{itemize}
	\item [Bridge] Un dispositivo bridge de software funciona exactamente igual que un bridge físico pero entre interfaces del mismo host. Es decir, copia frames de una a otra interfaz de capa 2 (reales o virtuales) efectuando filtrado por dirección MAC e inundando los broadcasts.
	\begin{itemize}
		\item Un bridge Linux \emph{esclaviza} a dos o más interfaces (reales o virtuales).
		\item Al ser incorporada una interfaz a un bridge, pierde su dirección IP y ésta pasa al bridge. 
		\item Los bridges Linux se controlan desde la línea de comandos con el comando brctl.
\end{itemize}
	\item[Ruteo] El kernel utiliza la tabla de ruteo para mover paquetes entre interfaces. Cuando aparece por una interfaz (real o virtual) un paquete de capa 3, se consulta la tabla de ruteo, que dice en cuál interfaz (real o virtual) debe copiarse.  
	\item [OpenSSL] Biblioteca de encriptación de uso general, para proteger la privacidad del tráfico a través de la VPN.
	\item [LZO] Biblioteca de compresión de datos para mejorar la performance del túnel creado por la VPN.
\end{description}



\subsection{Funcionamiento de OpenVPN en capa 2}

\begin{enumerate}
	\item En el host cliente:
	\begin{itemize}
		\item Se crea un dispositivo tapX.
		\item Se establece un bridge entre una interfaz local (ethX) y el tapX.
		\item La dirección IP de la interfaz local pasa al bridge.
		\item El proceso cliente de OpenVPN abre el dispositivo tapX y además establece conexión con el servidor a través de Internet.
		\item Cuando el host cliente reciba de su red local un frame por ethX, el bridge filtrará el frame hacia el tapX. El proceso cliente de OpenVPN recibirá el frame y lo derivará por la conexión TCP/IP hacia el servidor.
		\item Cuando el proceso cliente reciba un frame a través de la conexión TCP/IP, lo escribirá por la interfaz tapX. El bridge del host cliente recibirá el frame y le aplicará la acción de bridging correspondiente. 
	\end{itemize}
	\item En el host servidor:
	\begin{itemize}
		\item Se crea un dispositivo tapX.
		\item Se establece un bridge entre una interfaz local (ethX) y el tapX.
		\item La dirección IP de la interfaz local pasa al bridge.
		\item El proceso servidor de OpenVPN abre el dispositivo tapX y espera pedido de conexión del cliente a través de Internet.
		\item Cuando el host servidor reciba de su red local un frame por ethX, el bridge filtrará el frame hacia el tapX. El proceso servidor de OpenVPN lo tomará y lo derivará por la conexión TCP/IP hacia el cliente.
		\item Cuando el proceso servidor reciba un frame a través de la conexión TCP/IP, lo escribirá por la interfaz tapX. El bridge del host cliente recibirá el paquete y le aplicará la acción de bridging correspondiente. 
	\end{itemize}
\end{enumerate}

\subsection{Funcionamiento de OpenVPN en capa 3}

\begin{enumerate}
	\item En el host cliente:
	\begin{itemize}
		\item Se crea un dispositivo tunX.
		\item El proceso cliente de OpenVPN abre el dispositivo tunX y además establece conexión con el servidor a través de Internet.
		\item El tunX local recibe dirección IP A. Queda establecido un túnel entre procesos cliente y servidor cuyos extremos tienen direcciones IP A y B.
		\item Se establece una ruta que dirige los paquetes destinados a la red Y via tunX (próximo salto: B).
		\item Cuando el host cliente reciba de su red local, o de procesos locales, un paquete destinado a la red Y, lo escribirá en la interfaz de red virtual tunX. El proceso cliente de OpenVPN lo recibirá y lo derivará por la conexión TCP/IP hacia el servidor.
		\item Cuando el proceso cliente reciba un paquete a través de la conexión TCP/IP, lo escribirá por la interfaz tunX. El kernel del host cliente recibirá el paquete y le aplicará las reglas de ruteo correspondientes. 
	\end{itemize}
	\item En el host servidor:
	\begin{itemize}
		\item Se crea un dispositivo tunX.
		\item El proceso servidor de OpenVPN abre el dispositivo tunX y espera pedido de conexión del cliente a través de Internet.
		\item El tunX local recibe dirección IP B. Queda establecido un túnel entre procesos cliente y servidor cuyos extremos tienen direcciones IP A y B. 
		\item Se establece una ruta que dirige los paquetes destinados a la red X via tunX (próximo salto: A).
		\item Cuando el kernel del host servidor reciba de su red local, o de procesos locales, un paquete destinado a la red X, lo escribirá en la interfaz de red virtual tunX. El proceso servidor de OpenVPN tomará el paquete y lo derivará por la conexión TCP/IP hacia el cliente.
		\item Cuando el proceso servidor reciba un paquete a través de la conexión TCP/IP, lo escribirá por la interfaz tunX. El kernel del host servidor recibirá el paquete y le aplicará las reglas de ruteo correspondientes. 
	\end{itemize}
\end{enumerate}

\figura{vpnl2} {VPN de capa 2, trafica frames} {vpn-l2.jpg}

\figura{vpnl3} {VPN de capa 3, trafica paquetes} {vpn-l3.jpg}

\section{Configuración de OpenVPN}

\subsection{Autenticación mediante clave privada}
\begin{itemize}
	\item PKI = Public Key Infrastructure
	\item La seguridad en redes comprende problemas como la confidencialidad y la integridad de los mensajes, y la autenticación de los usuarios o dispositivos
	\item Clave simétrica, con secreto compartido, vs. Claves asimétricas
	\item Dificultad de la distribución de claves, solución: claves asimétricas
	\item Cada usuario tiene $K+$ = Clave pública, $K-$ = Clave privada
	\item Dos escenarios de uso:
	\begin{itemize}
	\item Cuando A encripta con la $K+$ de B:
	\begin{itemize}
		\item A envía a B en forma segura, nadie puede leer el mensaje
		\item Sólo B puede desencriptarlo con su $K-$
		\item Se asegura la confidencialidad
	\end{itemize}
	\item Cuando A encripta con su propia $K-$:
	\begin{itemize}
		\item Cualquiera puede leer el mensaje con la $K+$ de A
		\item Pero sólo A puede haberlo escrito, con su $K-$ (firma digital)
		\item Se aseguran la integridad y la autenticidad
	\end{itemize} 
	\end{itemize}
	\item El problema al utilizar la $K+$ de A que hemos recibido es \emph{asegurar que es realmente la de A}
	\item Se resuelve mediante autoridades de certificación (CA), entidades confiables
	\item La clave pública de A, junto con la identidad de A, firmada digitalmente por una CA, es un \emph{certificado} de A
	\item Conexión segurizada por PKI
	\begin{enumerate}
		\item Fase de autenticación mutua mediante intercambio de certificados, validando los certificados según la CA.
		\item Usando las $K+$ recibidas, las partes acuerdan en forma segura una clave simétrica que sirve para esta sesión.
		\item El resto de la comunicación se encripta usando esta clave simétrica (métodos DES, 3DES, AES).
	\end{enumerate} 
\end{itemize}


\subsection{Crear una Autoridad de Certificación}

En el servidor de VPN, copiar el directorio \lstinline{/usr/share/doc/openvpn/examples/easy-rsa} y todos sus contenidos a /etc/openvpn
\begin{lstlisting}
cd /etc/openvpn/easy-rsa/2.0
vi vars
. vars
./clean-all
./build-dh
./build-ca
./build-key-server servidor
./build-key cliente1
./build-key cliente2
# Migrar los certificados
# En el server:
# ca.crt ca.key dh1024.pem r3.crt r3.key
# 
# En el cliente:
# ca.crt r1.crt r1.key
\end{lstlisting}


\subsection{Documentación online}
\begin{itemize}
	\item Proyecto Lihuén\footnote{\url{http://lihuen.linti.unlp.edu.ar/index.php?title=Configurando_Redes_Privadas_Virtuales_con_OpenVPN}}
	\item OpenVPN HOWTO\footnote{\url{http://openvpn.net/index.php/open-source/documentation/howto.html}}
\end{itemize}


\subsection{Vinculación de LANs mediante Openvpn}
\begin{itemize}
	\item Instalar OpenVPN en los nodos de frontera de las LANs.
	\item Decidir cuál nodo será el servidor. Debe tener dirección IP pública accesible desde los clientes. No es necesaria la traducción DNS salvo que se trate de una IP dinámica.
	\item Preparar una autoridad de certificación (por ejemplo, en el servidor), y certificados para el servidor y para los clientes.
	\item Se pueden utilizar los modelos de archivos de configuración existentes en \url{/usr/share/doc/openvpn/examples}.
	\item En el directorio \lstinline{/etc/openvpn} del servidor preparar el archivo de configuración \lstinline{server.conf}, y en los clientes, \lstinline{client.conf}. Al arrancar, OpenVPN lee todos los archivos con extensión .conf que existan en ese directorio, y crea un proceso por cada uno, con la configuración que contengan.
	\item Modificar los parámetros de configuración:
	\begin{enumerate}
		\item Cliente o servidor
		\item Dirección y port, nodo local o remoto
		\item Protocolo UDP o TCP
		\item Device Tun o Tap, según la capa de operación de la VPN  
		\item Nombres de los archivos de clave y certificado 
		\item Archivo de log, útil para debugging de la configuración y operación
		\item Rutas a redes propias que deban ser inyectadas en el peer
\end{enumerate} 
	\item Arrancar o detener la VPN con el comando \lstinline{/etc/init.d/openvpn [start|stop]}.
\end{itemize}

\subsubsection {OpenVPN de nivel de Red}
Al configurar OpenVPN en capa 3, o nivel de Red, la configuración de OpenVPN puede asumir la creación de rutas entre las redes clientes y las de detrás del servidor. Estas rutas aparecen y desaparecen en la tabla de ruteo de los nodos extremos de la VPN según se activa o desactiva el proceso de OpenVPN.

\begin{itemize}
	\item Para que el cliente conozca las redes detrás del servidor, éste le inyecta las rutas correspondientes con la opción push de la configuración.
	\item El servidor instala rutas a las redes detrás de los clientes si se  especifican con la directiva route de la configuración del servidor. Además, para cada cliente, debe haber un archivo con la directiva iroute dentro del subdirectorio /etc/openvpn/ccd.
\end{itemize}

\subsubsection {OpenVPN de nivel de Enlace}
Al configurar OpenVPN en capa 2, o nivel de Enlace, la configuración debe incluir la administración del bridge entre la interfaz de red local y el dispositivo Tap que es utilizado por la VPN. Los frames que llegan a la interfaz de la red local serán copiados por el bridge en el Tap y seguirán su camino a través de la VPN.

Para la administración del bridge hay dos estrategias básicas: 
\begin{itemize}
	\item El bridge puede tener una definición estática, y ser siempre activado cada vez que arranca el equipo, y sólo ser desactivado cuando se detiene el equipo.
	\item El bridge puede ser activado y desactivado cuando se inicia y detiene OpenVPN. 
\end{itemize}


En el primer caso, la existencia del bridge es completamente independiente de la activación de la VPN. Para esta opción existe una forma de configuración del bridge que es dependiente del sistema operativo (ver Anexo \ref{subsec:staticbridge}). 

Para la segunda opción, es conveniente utilizar los scripts de creación de tap y montado de bridges que se muestran en el Anexo \ref{sec:bridgeupdown} y dispararlos con las directivas up y down de la configuración de OpenVPN. Por ejemplo:

\begin{lstlisting}
...
ca ca.crt
cert server.crt
key server.key
dh dh1024.pem
...
up /etc/openvpn/bridge-up
down /etc/openvpn/bridge-down
\end{lstlisting}


\section{Temas de práctica}
El laboratorio de la Fig. \ref{fig:vpn-lab1} se encuentra implementado sobre máquinas virtuales. 
\begin{comment}Utilizando una consola, encontrará en el directorio \lstinline{labs/openvpn-1} una configuración de laboratorio que se arranca con el comando \lstinline{lstart} y se detiene con \lstinline{lhalt}. 
\end{comment}


\figura{vpn-lab1}{Laboratorio 1 OpenVPN}{vpn-lab1.jpg}

El equipo virtual $R2$, que es gateway default de los otros routers, simula la Internet, en el sentido de que descarta todo tráfico dirigido a redes privadas. Por este motivo las redes con prefijo 10.0.0.0/8, del laboratorio, no son accesibles una desde la otra, siendo necesaria una solución de Red Privada Virtual. La configuración de este equipo $R2$ no debe ser modificada.

\begin{itemize}
	\item Verifique que las direcciones, ruteo default y topología corresponden a la figura.
	\item Verifique servicio de nombres, ping y traceroute a www.google.com.
	\item Ping a los routers de la topología.
	\item Ping a los hosts de la misma LAN.
	\item Ping a los hosts de la LAN opuesta. 
	\item Compruebe navegación en ambiente de caracteres con el comando \texttt{lynx http://www.google.com}.
	\item Compruebe que puede correr el web server Apache en cualquiera de las PCs. 
	\item Compruebe navegación en ambiente de caracteres desde una PC al servidor Apache del otro nodo de la misma LAN.
\end{itemize}

Una vez que esté familiarizado con el escenario:
\begin{enumerate}
	\item Implementar una VPN de capa 3 entre ambas LANs. Verifique las interfaces existentes en los nodos cabecera y el contenido de sus tablas de ruteo. Comprobar que los  clientes de una de las redes pueden utilizar recursos (ssh, servidor http) de clientes de la otra. Instale servicios basados en broadcast (como Samba, sobre el protocolo SMB) y observe desde qué lugares pueden accederse.
	\item Modificar el direccionamiento IP del laboratorio para que todos los clientes reciban direcciones en la misma subred e implementar una VPN de capa 2. Comprobar que ambas redes forman un único dominio de broadcast. Verifique el comportamiento de los protocolos basados en broadcast como SMB. 
	\item ¿Usaría transporte UDP o TCP para una VPN en capa 2?
\end{enumerate}


\newpage
\part {Balance de carga y Alta Disponibilidad en redes}
%\section{Bonding}


Acoplamiento de dos o más interfaces de red, creando una interfaz virtual capaz de funcionar en diferentes modos. Se conoce con diferentes nombres (\textit{channel bonding, teaming, link aggregation}). 
Los modos de configuración de bonding definen el comportamiento del conjunto de interfaces y cumplen con diferentes objetivos. Algunos modos proporcionan tolerancia a fallos mediante redundancia; otros aumentan el ancho de banda disponible por agregación de enlaces.

Los modos active-backup, balance-tlb, y balance-alb no requieren una configuración especial en los switches a los cuales está conectado el bond.
Sin embargo, los modos 802.3ad, balance-rr, balance-xor y broadcast requieren capacidades especiales del switch definidas en documentos IEEE.


\begin{description}
\item [Modo 0 (balance-rr)]
Este modo transmite frames en orden secuencial desde el primer esclavo disponible hasta el último. Si un bond tiene dos interfaces reales, y llegan simultáneamente dos frames a ser enviados desde la interfaz bond, el primero será transmitido por el primer esclavo; el segundo frame, por el segundo esclavo; el tercer frame será transmitido por el primer esclavo, etc. Esto provee a la vez balance de carga y tolerancia a fallos. 

\item [Modo 1 (active-backup)]
Este modo coloca a las interfaces esclavas en estado de backup, y sólo se activará una de ellas si se pierde el link de la interfaz activa. En este modo, sólo hay un esclavo activo en el bond en cada momento. Sólo se activa un esclavo diferente si falla el esclavo activo. Este modo provee tolerancia a fallos.

\item [Modo 2 (balance-xor)]
Transmite basándose en una fórmula XOR. Se computa la operación XOR entre la dirección MAC de origen y la de destino, módulo la cantidad de esclavos (es decir, se toma el resto de dividir por la cantidad de esclavos). Este procedimiento tiene el efecto de seleccionar siempre el mismo esclavo para cada dirección MAC destino. Provee balance de carga y tolerancia a fallos.

\item [Modo 3 (broadcast)]
Este modo transmite todos los frames por todas las interfaces esclavas. Es el menos usado, sólo para propósitos específicos, y sólo provee tolerancia a fallos. 

\item [Modo 4 (802.3ad)]
Este modo se conoce como el modo de agregación dinámica de enlaces (Dynamic Link Aggregation). Crea grupos de agregación que comparten la misma velocidad y modos de duplexing. Este modo requiere un switch que soporte la norma IEEE 802.3ad (Dynamic Link).

\item [Modo 5 (balance-tlb)]
Llamado balance de carga adaptativo en transmisión. El tráfico de salida se distribuye de acuerdo a la carga actual y es encolado en cada interfaz esclava. El tráfico entrante es recibido por el esclavo actual. 

\item [Modo 6 (balance-alb)]
Este modo es el de balance de carga adaptativo. Esto incluye balance-tlb y balance de carga en recepción (rlb) para tráfico IPv4. El balance de carga en recepción se logra por negociación ARP. El driver de bonding intercepta las respuestas ARP enviadas por el servidor y sobreescribe la dirección MAC origen con la dirección MAC única de uno de los esclavos en el bond, de forma que diferentes clientes usen diferentes direcciones MAC para dirigirse al server. 
\end{description}

\subsection {Detección de eventos}
Los modos que ofrecen tolerancia a fallos necesitan algún mecanismo para detectar eventos de caída de las interfaces de red (NIC) locales, los enlaces, o las NICs de los extremos opuestos de los vínculos. Ante la detección de un evento de fallo, el bond fuerza la conmutación a otra interfaz, que entonces se convierte en primaria. Esta acción se llama \emph{failover}.  

Para la detección de eventos hay dos opciones posibles.

\begin{enumerate}
		\item MII (Medium Independent Interface). Especificación que cumplen la mayoría de las NICs modernas, que presenta datos de link activo o inactivo en forma independiente de la implementación del medio conectado. Se debe especificar los parámetros \lstinline$bond-miimon$ (intervalo de revisión del estado del link), \lstinline$bond-downdelay$ (tiempo desde que se detecta fallo hasta que se da de baja la interfaz) y \lstinline$bond-updelay$ (tiempo para volver a poner en servicio la interfaz una vez que vuelve el link a estado activo). 
		\item ARP. Se establece por configuración una cantidad de direcciones IP de control en la red local, y el bond emite consultas ARP periódicas a estas direcciones. Cuando se deja de recibir respuesta por la interfaz activa durante una cantidad de tiempo, configurable, se considera que ha caído el enlace y se realiza el failover. Se deben configurar los parámetros \lstinline$bond-arp-interval$ (intervalo entre emisión de ARP) y \lstinline$bond-arp-ip-target$ (lista de IPs confiables).
\end{enumerate}
  
\subsection {Configuración en Debian}

\subsubsection {Configuración con MII}
\begin{lstlisting}
# /etc/network/interfaces
# The loopback network interface
auto lo
iface lo inet loopback
auto eth0
iface eth0 inet manual
	bond-master bond0
	bond-primary eth0 eth1

auto eth1
iface eth1 inet manual
	bond-master bond0
	bond-primary eth0 eth1

auto bond0
iface bond0 inet static
	address 192.168.1.15
	netmask 255.255.255.0
	network 192.168.1.0
	gateway 192.168.1.1
	bond-slaves eth0 eth1
	bond-mode active-backup
	bond-miimon 100
	bond-downdelay 200
	bond-updelay 200

\end{lstlisting}

\subsubsection {Configuración con ARP}

\begin{lstlisting}
# /etc/network/interfaces
# The loopback network interface
auto lo
iface lo inet loopback

# No se especifican las auto ethX

auto bond0
iface bond0 inet static
	address 10.1.1.1
	netmask 255.255.255.0
	network 10.1.1.0
	bond_primary eth0
	slaves eth0 eth1
	bond-mode active-backup
	bond-arp-interval 2000
	bond-arp-ip-target 10.1.1.2 10.1.1.3

\end{lstlisting}



\subsection{Temas de práctica}

La topología de la Fig. \ref{fig:bonding} comprende un nodo llamado switch, y tres nodos conectados a él. El switch tiene cinco interfaces, cada una conectada a un enlace diferente. Dos de los nodos tienen dos interfaces cada uno, y el tercero una sola. Cada interfaz de los nodos host1 a host3 está conectada a su propio enlace. Los enlaces se denominan, de izquierda a derecha en el diagrama, link1 a link5. 

\figura[10]{bonding}{Configuración del laboratorio de bonding}{bonding.png}

\recuadro{

¡Ver las Notas sobre el laboratorio en la sección más abajo!

}


\begin{enumerate}
	\item Configure en forma estática la red del nodo denominado switch implementando un bridge que esclavice a todas sus interfaces. Dé direcciones en el mismo dominio de broadcast a los demás nodos. Compruebe llegada por ping. Observe el tráfico que pasa por las interfaces del switch con el comando tcpdump -i ethX. El nodo virtual switch, ¿resulta un buen modelo de un switch de hardware? ¿Necesita tener una dirección IP? 
	\item En los nodos host1 y host2, Instale el módulo que permite esclavizar las interfaces con \code{dpkg -i /hostlab/ifenslave-...deb}.
	\item Configure el nodo host1 indicando en /etc/modules que debe ser cargado el módulo bonding al arranque. Configure la red del mismo nodo, ligando ambas interfaces mediante un bond en modo active-backup. ¿Qué capacidades tiene este modo?
	\item Mantenga un ping desde host3 a host1, observando el tráfico por las interfaces del switch. Simule la caída del vínculo con host1.  ¿Se interrumpe el ping? ¿Sigue pasando el tráfico por las mismas interfaces?
	\item Configure el nodo host2 del mismo modo que en el punto 3, pero con el bond en modo balance-rr. ¿Qué capacidades tiene este modo?
	\item Ejecute el mismo experimento del punto 4 pero desde host3 a host2. 
\end{enumerate}


\subsubsection{Notas sobre el laboratorio}
\begin{itemize}
	\item Es necesario instalar el paquete \code{ifenslave}. En nuestro laboratorio se ha provisto un ejemplar del archivo DEB correspondiente en el directorio \code{/hostlab}.
	\item Es necesario indicar el modo, técnica de detección de eventos, y parámetros, en el archivo \code{/etc/modules} (a pesar de indicarlo en \code{/etc/network/interfaces}), de la siguiente manera:
	\begin{lstlisting}
	alias bond0 bonding
	options bonding mode=1 miimon=100 downdelay=200 updelay=200
	\end{lstlisting}
(detección por MII) o bien:
	\begin{lstlisting}
	alias bond0 bonding
	options bonding mode=1 arp_interval=1000 arp_ip_target=10.1.1.1
	\end{lstlisting}
(detección por ARP).
	\item Al probar sistemas de Alta Disponibilidad, una parte muy importante es la inyección de fallas. La tecnología de virtualización utilizada para el laboratorio no permite replicar correctamente el evento de caída de interfaz, enlace o peer, cuando se usa la técnica de detección de fallas MII (la interfaz aparece siempre conectada). La mejor aproximación que hemos logrado a la simulación de la falla es cuando se elige detección de fallas por ARP.
	\item Para simular la caída de un vínculo se sugiere detener el proceso UML que simula el enlace. Logramos esto buscando entre todos los procesos \code{uml_switch} de la máquina host aquel relacionado con el nombre del enlace. Por ejemplo, para simular la caída del link1:
	\begin{lstlisting}
# ps f | grep uml_switch
23137 pts/1    S+     0:00  \_ grep uml_switch
...
19381 pts/1    S      0:00 /home/rii/netkit/bin/uml_switch -hub -unix /home/oso/.netkit/hubs/vhub_oso_link1.cnct
# kill -STOP 19381
\end{lstlisting}
	El proceso se hace continuar (se restablece el vínculo virtual) con \code{kill -CONT 19381}.
	\item Cada vez que hagamos un cambio de configuración será preferible, antes de hacer una nueva prueba, bajar y volver a levantar el laboratorio completo con \code{lhalt} y \code{lstart}. 
	\item La interfaz virtual bond puede monitorearse mirando el pseudo archivo correspondiente en el directorio /proc:
\begin{lstlisting}
# cat /proc/net/bonding/bond0 
Ethernet Channel Bonding Driver: v3.2.5 (March 21, 2008)

Bonding Mode: fault-tolerance (active-backup)
Primary Slave: eth0
Currently Active Slave: eth0
MII Status: up
MII Polling Interval (ms): 0
Up Delay (ms): 0
Down Delay (ms): 0
ARP Polling Interval (ms): 2000
ARP IP target/s (n.n.n.n form): 10.1.1.2, 10.1.1.3

Slave Interface: eth0
MII Status: up
Link Failure Count: 0
Permanent HW addr: a6:8a:d7:63:1e:49

Slave Interface: eth1
MII Status: up
Link Failure Count: 0
Permanent HW addr: e6:c9:c8:a3:eb:72
\end{lstlisting}
\end{itemize}



\subsection{Referencias}
\begin{itemize}
	\item \url{https://www.kernel.org/doc/Documentation/networking/bonding.txt}
	\item \url{http://www.linuxfoundation.org/collaborate/workgroups/networking/bonding}
	\item \url{http://www.cyberciti.biz/tips/debian-ubuntu-teaming-aggregating-multiple-network-connections.html}
\end{itemize}







%----------- A N E X O S ---------
\newpage
\part {Anexos}
\appendix
\subsection{iptables.log}
\label{subsec:iptables.log}
\begin{lstlisting}
 Logged 539 packets on interface eth1
   From 0000:0000:1011:1213:0100:0000:0000:0000 - 3 packets to icmpv6(130)
   From 0000:0000:0000:859e:0100:0000:0000:0000 - 1 packet to icmpv6(130)
   From 0000:0023:ff53:4d42:0100:0000:0000:0000 - 1 packet to icmpv6(130)
   From 0000:0000:0000:3433:0100:0000:0000:0000 - 1 packet to icmpv6(130)
   From 0000:0000:0000:3132:0100:0000:0000:0000 - 1 packet to icmpv6(130)
   From 0000:0000:0000:7d3a:0100:0000:0000:0000 - 1 packet to icmpv6(130)
   From 0000:0000:0000:6e63:0100:0000:0000:0000 - 1 packet to icmpv6(130)
   From 0000:0000:0000:937f:0100:0000:0000:0000 - 1 packet to icmpv6(130)
   From 0000:0000:0000:0000:0000:0000:0000:0000 - 2 packets to icmpv6(130)
   From 0000:0000:0000:0000:0100:0000:0000:0000 - 40 packets to icmpv6(130)
   From 0000:0000:0000:6569:0100:0000:0000:0000 - 1 packet to icmpv6(130)
   From 000e:175f:531c:580e:0100:0000:0000:0000 - 1 packet to icmpv6(130)
   From 0011:11db:a2d4:0a00:0100:0000:0000:0000 - 2 packets to icmpv6(130)
   From 002c:799a:0694:8c26:0100:0000:0000:0000 - 1 packet to icmpv6(130)
   From 0100:0000:0600:0000:0100:0000:0000:0000 - 2 packets to icmpv6(130)
   From 0101:080a:00c6:0621:0100:0000:0000:0000 - 1 packet to icmpv6(130)
   From 0101:080a:00c6:f565:0007:994d:0000:0000 - 1 packet to icmpv6(130)
   From 0101:080a:00c7:39ad:0100:0000:0000:0000 - 1 packet to icmpv6(130)
   From 0101:080a:00c7:3e49:0100:0000:0000:0000 - 1 packet to icmpv6(130)
   From 0101:080a:00c7:5bd1:0100:0000:0000:0000 - 1 packet to icmpv6(130)
   From 0101:080a:1551:7183:0100:0000:0000:0000 - 1 packet to icmpv6(130)
   From 0101:080a:1a9a:1543:0100:0000:0000:0000 - 1 packet to icmpv6(130)
   From 0101:080a:4553:94db:0100:0000:0000:0000 - 1 packet to icmpv6(130)
   From 0101:080a:4557:126c:0100:0000:0000:0000 - 1 packet to icmpv6(130)
   From 0101:080a:4559:6ffd:0100:0000:0000:0000 - 1 packet to icmpv6(130)
   From 0101:080a:4559:e9ac:0100:0000:0000:0000 - 1 packet to icmpv6(130)
   From 0101:080a:455a:3fd8:0100:0000:0000:0000 - 1 packet to icmpv6(130)
   From 0101:080a:455a:cc78:0100:0000:0000:0000 - 1 packet to icmpv6(130)
   From 0101:080a:455c:c658:0100:0000:0000:0000 - 1 packet to icmpv6(130)
   From 0101:080a:455d:4147:0100:0000:0000:0000 - 1 packet to icmpv6(130)
   From 0101:080a:455d:bbfc:0100:0000:0000:0000 - 1 packet to icmpv6(130)
   From 0101:080a:455e:3351:0100:0000:0000:0000 - 1 packet to icmpv6(130)
   From 0101:080a:4567:104c:0100:0000:0000:0000 - 1 packet to icmpv6(130)
   From 0101:080a:4575:8fa3:0100:0000:0000:0000 - 1 packet to icmpv6(130)
   From 0101:080a:4575:fcd3:0100:0000:0000:0000 - 1 packet to icmpv6(130)
   From 0101:080a:4584:d388:0100:0000:0000:0000 - 1 packet to icmpv6(130)
   From 0101:080a:458c:f368:0100:0000:0000:0000 - 1 packet to icmpv6(130)
   From 0101:080a:4594:8809:0100:0000:0000:0000 - 1 packet to icmpv6(130)
   From 0102:0417:0000:0000:0100:0000:0000:0000 - 1 packet to icmpv6(130)
   From 0102:a16d:0a00:00c8:0100:0000:0000:0000 - 1 packet to icmpv6(130)
   From 0204:05b4:0402:080a:0100:0000:0000:0000 - 2 packets to icmpv6(130)
   From 0300:0000:0400:0000:0100:0000:0000:0000 - 3 packets to icmpv6(130)
   From 036b:696d:0675:6e63:0100:0000:0000:0000 - 1 packet to icmpv6(130)
   From 10d4:d5cb:f80c:14d5:0100:0000:0000:0000 - 1 packet to icmpv6(130)
   From 1400:0300:9709:0000:0100:0000:0000:0000 - 1 packet to icmpv6(130)
   From 2269:6422:3a20:2232:0100:0000:0000:0000 - 1 packet to icmpv6(130)
   From 3037:3337:3431:3832:0100:0000:0000:0000 - 1 packet to icmpv6(130)
   From 3135:3332:3a38:3439:0100:0000:0000:0000 - 1 packet to icmpv6(130)
   From 3234:3238:323a:3834:0100:0000:0000:0000 - 1 packet to icmpv6(130)
   From 3333:3238:323a:3834:0100:0000:0000:0000 - 1 packet to icmpv6(130)
   From 3335:3839:323a:3834:0100:0000:0000:0000 - 1 packet to icmpv6(130)
   From 3336:3532:323a:3834:0100:0000:0000:0000 - 1 packet to icmpv6(130)
   From 3336:3837:3039:3132:0100:0000:0000:0000 - 8 packets to icmpv6(130)
   From 3430:3734:3330:3532:0100:0000:0000:0000 - 1 packet to icmpv6(130)
   From 3432:3230:3239:3a33:0100:0000:0000:0000 - 1 packet to icmpv6(130)
   From 3432:3635:3239:3a33:0100:0000:0000:0000 - 1 packet to icmpv6(130)
   From 3433:3230:3332:3a38:0100:0000:0000:0000 - 1 packet to icmpv6(130)
   From 3433:3534:3039:3a33:0100:0000:0000:0000 - 1 packet to icmpv6(130)
   From 3734:3237:3339:313a:0100:0000:0000:0000 - 1 packet to icmpv6(130)
   From 3734:3330:3238:313a:0100:0000:0000:0000 - 1 packet to icmpv6(130)
   From 3734:3330:3636:313a:0100:0000:0000:0000 - 1 packet to icmpv6(130)
   From 3734:3330:3930:323a:0100:0000:0000:0000 - 1 packet to icmpv6(130)
   From 3734:3334:3533:313a:0100:0000:0000:0000 - 1 packet to icmpv6(130)
   From 3734:3334:3736:393a:0100:0000:0000:0000 - 1 packet to icmpv6(130)
   From 616a:6f72:223a:2031:0100:0000:0000:0000 - 1 packet to icmpv6(130)
   From 6d65:6e74:7322:3a7b:0100:0000:0000:0000 - 9 packets to icmpv6(130)
   From 6f20:3134:3037:3433:0100:0000:0000:0000 - 2 packets to icmpv6(130)
   From 7473:223a:7b7d:7d00:0100:0000:0000:0000 - 1 packet to icmpv6(130)
   From 7473:223a:7b7d:7d32:0100:0000:0000:0000 - 1 packet to icmpv6(130)
   From 7473:223a:7b7d:7d3a:0100:0000:0000:0000 - 4 packets to icmpv6(130)
   From 7473:223a:7b7d:7d7b:0100:0000:0000:0000 - 3 packets to icmpv6(130)
   From 756e:6e69:6e67:223a:0100:0000:0000:0000 - 1 packet to icmpv6(130)
   From bf1d:f661:984e:fcb0:0100:0000:0000:0000 - 1 packet to icmpv6(130)
   From c011:fdda:43fe:4d59:65fe:e1aa:c7d5:683a - 1 packet to icmpv6(130)
   From c012:aa5c:173c:261c:0100:0000:0000:0000 - 1 packet to icmpv6(130)
   From c012:cfa3:e641:3b5f:0100:0000:0000:0000 - 1 packet to icmpv6(130)
   From c013:4b15:1c15:3311:0100:0000:0000:0000 - 1 packet to icmpv6(130)
   From c013:9389:bcf8:f7d4:0100:0000:0000:0000 - 1 packet to icmpv6(130)
   From c014:ff7d:0000:0000:0100:0000:0000:0000 - 1 packet to icmpv6(130)
   From e063:837f:0000:0000:0100:0000:0000:0000 - 3 packets to icmpv6(130)
   From e063:837f:2077:937f:0100:0000:0000:0000 - 1 packet to icmpv6(130)
   From e063:837f:301f:937f:0100:0000:0000:0000 - 3 packets to icmpv6(130)
   From 0.0.0.0 - 154 packets to igmp(0)
   From 10.0.3.4 - 154 packets to igmp(0)
   From 10.0.3.21 - 75 packets to igmp(0)
   From 10.0.3.209 - 2 packets to igmp(0)

 Listed by source hosts:
 Logged 18 packets on interface virbr0
   From fe80:0000:0000:0000:5054:00ff:feed:8246 - 18 packets to udp(5353)

 Listed by source hosts:
 Logged 50 packets on interface wlan0
   From 10.0.4.1 - 2 packets to udp(68)
   From 64.233.186.188 - 15 packets to tcp(44421,53418)
   From 64.235.151.8 - 5 packets to tcp(56214)
   From 173.194.42.0 - 1 packet to tcp(56677)
   From 173.194.42.21 - 3 packets to tcp(58597)
   From 173.194.42.22 - 7 packets to tcp(35536)
   From 173.194.42.75 - 3 packets to tcp(48585)
   From 173.194.42.85 - 4 packets to tcp(44550)
   From 173.194.42.86 - 1 packet to tcp(51940)
   From 192.168.1.1 - 2 packets to udp(68)
   From 192.168.2.1 - 2 packets to udp(68)
   From 195.135.221.134 - 1 packet to tcp(51756)
   From 195.154.174.66 - 1 packet to tcp(58047)
   From 200.42.136.212 - 3 packets to tcp(59351,59361)
\end{lstlisting}

%------------------------------------------------------------

\end{document}